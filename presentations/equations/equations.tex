% This text is proprietary.
% It's a part of presentation made by myself.
% It may not used commercial.
% The noncommercial use such as private and study is free
% Sep. 2005 
% Author: Sascha Frank 
% University Freiburg 
% www.informatik.uni-freiburg.de/~frank/

\documentclass{beamer}
%\usepackage[usenames]{color} %used for font color
\renewcommand{\familydefault}{\sfdefault}
\usepackage{xcolor,transparent,tikz}
\usepackage{amsthm,amsfonts,mathtools,amssymb,amsmath,bm} %maths
\usepackage[utf8]{inputenc} %useful to type directly diacritic characters
\definecolor{dr}{RGB}{181,23,0}
\definecolor{db}{RGB}{0,31,164}
% commands
\newcommand{\unc}{\color{dr}}
\newcommand{\secstage}{\color{db}}
\newcommand{\rxi}{{\unc \bm{\xi}}}
\newcommand{\rxihat}[1]{{\unc \hat{\bm{\xi}}_{#1}}}
\newcommand{\by}{{\secstage y}}

\newcommand{\y}{{\color{db}\bm{y}}}
\newcommand{\x}{{\color{black}\bm{w}}}
\newcommand{\X}{W}
\newcommand{\Y}{K}
\newcommand{\expectationQ}[1]{%
    \begingroup
    \colorlet{savedleftcolor}{.}
    \unc\mathbb{E}_\mathbb{Q}\left[\color{savedleftcolor}#1\unc\right]
    \endgroup
}
\newcommand{\expectation}[1]{%
    \begingroup
    \colorlet{savedleftcolor}{.}
    \unc\mathbb{E}_\mathbb{P}\left[\color{savedleftcolor}#1\unc\right]
    \endgroup
}
\newcommand{\expectationNominal}[1]{%
    \begingroup
    \colorlet{savedleftcolor}{.}
    \unc\mathbb{E}_{\hat{\mathbb{P}}_N}\left[\color{savedleftcolor}#1\unc\right]
    \endgroup
}
\newcommand{\loss}[1]{%
    \begingroup
    \colorlet{savedleftcolor}{.}
    \secstage\ell\left(\color{savedleftcolor}#1\secstage\right)
    \endgroup
}
\newcommand{\myphi}[1]{%
    \begingroup
    \colorlet{savedleftcolor}{.}
    \secstage\phi\left(\color{savedleftcolor}#1\secstage\right)
    \endgroup
}

\begin{document}
\title{Simple Beamer Class}   
\author{Sascha Frank} 
\date{\today} 

\frame{\titlepage} 


\section{Optimization under uncertainty} 
\frame{\frametitle{Optimization under uncertainty} 
\begin{equation*}
\begin{array}{r@{\;\;}l@{\;}l}
\mathop{\text{minimize}} \;\; & {\color{db} f_1(x_1)} &+ {\color{dr} f_2(x_2)} \\
\text{subject to} \;\; & \color{db} h_1(x_1) \leq 0,  &\phantom{+} \color{dr} h_2(x_2) \;\leq 0 \\
& \color{db} {A}_1 x_1 &+ \color{dr} {A}_2 x_2 \color{black} = 0
\end{array}
\end{equation*}

\begin{equation*}
\color{db} x_1 = P_g, Q_g \text{ of \tikz\draw[fill] (0,0) circle (0.75ex); and } V, \theta \text{ of \tikz\draw[fill] (0,0) circle (0.75ex); + \tikz\draw[dashed] (0,0) circle (0.75ex); }
\end{equation*}

\begin{equation*}
\color{dr} x_2 = P_g, Q_g \text{ of \tikz\draw[fill] (0,0) circle (0.75ex); and } V, \theta \text{ of \tikz\draw[fill] (0,0) circle (0.75ex); + \tikz\draw[dashed] (0,0) circle (0.75ex); }
\end{equation*}
}

\frame{
\begin{equation*}
\color{db} t = 1
\end{equation*}	

\begin{equation*}
\color{dr} t = 2
\end{equation*}

\begin{equation*}
\left\lvert {\color{db} P_g} - {\color{dr} P_g} \right\rvert \leq \Delta
\end{equation*}
}
\end{document}

